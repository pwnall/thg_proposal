\section{Design Approach}
\label{sec:design}

Our system aims to offer the same integrity guarantees as SGX, but also provide
full privacy for the code executing inside a protected container, by hiding the
code's memory access patterns against the attacks described in \S
\ref{sec:sgx_leaks}.

We rely on a protected container that is very similar to an SGX enclave. We
pursue two avenues for designing the container. We will explore a minimal set
of SGX modifications that would prevent against memory access patterns leaking
from enclave execution. We will also explore a clean-slate design that
specifies a minimal set of additions to a RISC architecture resulting in a
protected contained that meets our threat model.

In our system, the infrastructure's kernel or hypervisor is responsible for
creating the protected container and loading our trusted runtime into it. The
runtime loads the untrusted software and private data, and verifies that the
protected container was set up correctly, and that the untrusted software
only interacts with the outside world through the runtime. For example, all
the software's memory access instructions must be replaced by calls into our
runtime's memory manager, which that implements an oblivious RAM (ORAM)
protocol \cite{stefanov2012path} that hides the software's memory access
patterns.

SGX has an attestation mechanism that uses a signature of the enclave's initial
contents to set up a secure communication channel between a remote user and
the software running inside the enclave. In our system, the signature only
covers our trusted runtime, reflecting the fact that the software computing on
the private data is not in the TCB.


\subsection{The Memory Manager}

The memory manager sets up the L2 cache as a trusted scratchpad. We target the
L2 cache because we trust all the components on the CPU die, and the L2 cache
is the largest on-die cache where we understand the set indexing method
(\S \ref{sec:caching}).

The memory manager partitions the sets of the L2 cache into scratchpad sets and
real memory sets. The untrusted software's code and data must only occupy
addresses that map to the real memory sets. The protected environment must also
have a memory area that maps to the scratchpad sets.

The scratchpad is used like a cache for the untrusted software's data. The
memory manager provides functions for reading and writing a memory location.
If the desired location is cached in the scratchpad, it is read or written
immediately. Otherwise, a cache line is evicted from the scratchpad, and
replaced with the line that contains the desired location. If evicted cache
line is dirty, it is written to the software's memory using an ORAM protocol.

To avoid leaking information via memory access timing, the memory manager
runs the ORAM protocol periodically, using the \texttt{RDTSC} instruction as
the clock source. On a read, the memory manager spins in a loop until it can
run the ORAM protocol. Writes are queued up, and the memory manager only spins
in a loop if the queue is full. The untrusted software must also periodically
call into our runtime, so it can run the ORAM protocol at the right time
even if no memory accesses are pending.

To remain future-proof, the memory manager uses the \texttt{CPUID} instruction
\cite{intel2013manual} to query the CPU's cache layout and aborts when faced
with an implementation that doesn't match our design assumptions.


\subsection{The Loader and Verifier}

We follow the approach of Google Native Client \cite{yee2009native}
\cite{sehr2010adapting}, namely we require that the untrusted software
satisfies some constraints that are easy to verify, and that guarantee that the
software will always interact with the outside world through our trusted
runtime. We provide a compiler that produces code meeting these constraints,
but the compiler is untrusted, and the software provider is free to replace our
compiler with any tool that produces code meeting our constraints. This results
in a small TCB, compared to rewriting the software's machine code on the fly,
and we expect that a technique similar to RockSalt \cite{morrisett2012rocksalt}
can be used to prove the correctness of our verifier.

The main constraint in our system is that untrusted software cannot contain
memory access instructions. All memory accesses must be performed by calling
into our memory manager. The software must also call into our memory manager
at the beginning of every basic block. Long basic blocks must call into our
runtime every 100\footnote{Subject to tweaking} instructions.

Our runtime hides the untrusted software's actual running time, to remove
information leak. Instead, the software's metadata specifies a deadline.
If the software does not meet the deadline, it is terminated. If it does, the
runtime spins until the deadline is reached. To implement this, we prohibit the
untrusted software from using \texttt{EEXIT} directly, and require that it
calls into our runtime instead. The periodic calls into the runtime metioned
above terminate the untrusted software if its deadline has passed.


\subsection{The Protected Environment}

SGX offers a protected environment for security-sensitive software, but the
environment leaks the memory access patterns (\S \ref{sec:sgx_leaks}). The
design described here needs stronger guarantees to prevent the attacks afforded
by our threat model.

The x86 address translation (\S \ref{sec:paging}) gives malicious system
software an opportunity to snoop on the enclave software, because of the
information reported by page faults. System software can also foil the cache
partitioning scheme used by our memory manager, because it controls some of the
bits in the physical memory addresses used to compute the cache set index. We
are exploring mechanisms for constraining the mapping of an enclave page to an
EPC page. The extra constraints would only be checked when an enclave page is
created (via \texttt{EADD}) or loaded into the EPC (via \texttt{ELDB}).

Asynchronous Enclave Exits (AEX) give malicious system software an opportunity
to preempt the enclave software and mount cache timing attacks. We are
investigating servicing faults inside the enclave, instead of performing an
AEX, and asking the local APIC to route interrupts to a different CPU if the
current CPU is executing enclave code. This approach would require the enclave
code to \texttt{EEXIT} in a timely manner, so the system software can schedule
other threads on the CPU. The deadline-enforcing mechanism in our runtime would
ensure that the untrusted software \texttt{EEXIT}s in time. The CPU would
prevent against malicious or buggy enclave code that loops forever by
implementing an inter-processor-interupt (IPI) that the system software could
issue from another CPU core to destroy the unresponsive enclave and flush the
core's caches.

AEX removal would make it impossible for the system software to take advantage
of hyper-threading to snoop on the enclave software by scheduling a malicious
thread on a logical processor in the same core as the logical processor
executing the enclave software. Our runtime would spin loop at start-up, until
the system software schedules enclave threads on all the logical processors
inside the core used to run enclave software. A simple approach would use one
thread to run enclave code, and spin loop on all the other threads. We are
also considering dedicating a logical processor to the ORAM protocol, and
another logical processor to running the untrusted software.
