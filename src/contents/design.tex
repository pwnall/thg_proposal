\section{Design}
\label{sec:design}


We avoid leaking memory access patterns by replacing all the instructions that
perform memory accesses in the software operating on private data with calls
to a memory manager that is part of our TCB. The memory manager implements an
oblivious RAM protocol \cite{stefanov2012path} to hide the software's memory
access patterns. We follow the approach of Google Native Client
\cite{yee2009native} \cite{sehr2010adapting}, namely we require that memory
accesses are replaced by calls to a function during the software's compilation
phase, and our TCB contains a loader that statically verifies the software to
ensure that it does not perform any direct memory access. This results in a
small TCB, compared to rewriting the software's machine code on the fly, and
we expect that a technique similar to RockSalt \cite{morrisett2012rocksalt} can
be used to prove the correctness of our verifier.

SGX provides an attestation mechanism that assures a remote user that a piece
of information was signed by a certain piece of software running inside a
enclave. The signature includes a cryptographic hash of the software running
inside the enclave. In our system, the signature only needs to cover the loader
and memory manager, as the loader verifies the program that computes on
private data, and ensures that the program meets the data owner's security
policy. The data owner must only trust our TCB, and the software provider can
iterate faster and rely on our verifier and memory manager to protect from
bugs that would result in private data leaks.


