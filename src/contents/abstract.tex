\begin{abstract}

We set out to build a system for performing trusted computation on mainstream
desktop and server-class computers. Our system has three mutually distrusting
parties: \textit{a data provider} has some private information, and allows
\textit{a software provider} to perform some computation on the private data,
using hardware resources provided by \textit{an infrastructure owner}. The
computation result is communicated to either the data provider or the software
provider, in encrypted form.

We aim to build a system that can prevent malicious software from colluding
with either malicious infrastructure, or with another piece of malicious
software running on the same infrastructure, for the purpose of leaking private
data that is not included in the computation result. Our trusted computing base
(TCB) includes the contents of an Intel processor, and a
\textit{software loader} that verifies and launches the untrusted software. We
rely on a processor's ability to create a secure execution environment that is
isolated from the system software, such as Software Guard Extensions (SGX)
\cite{mckeen2013innovative} \cite{anati2013sgx} or Trusted Execution Technology
(TXT) \cite{grawrock2009txt}. Our TCB does not include the operating system or
any high-level software that runs on the infrastructure.

The main application that we are targeting is enabling heavily regulated
fields, such as the medical and financial industries, to use public clouds and
software as a service (SaaS). While legal consequences make it unlikely that a
public infrastructure or SaaS provider will deliberately act in a malicious
manner, our threat model covers against skilled attackers who can exploit bugs
in both the infrastructure and the software that performs computation over the
private data.

\end{abstract}
