\section{Implementation Plan}
\label{sec:implementation}

We will implement the Memory Manager and the Loader and Verifier described in
\S \ref{sec:design}, in order to assess their performance impact and the amount
of code required in the TCB. Our prototype will not have the security
properties required by our threat model, due to missing hardware support.

Concretely, we will build the following components:

\begin{itemize}

\item A \textit{Loader} (\S \ref{sec:loader_verifier}) that sets up a process
      to become a host for a Protected Environment
      (\S \ref{sec:protected_environment}).

\item A \textit{Verifier} (\S \ref{sec:loader_verifier}) that checks whether a
      piece of untrusted code meets our requirements for being executed inside
      a Protected Environment. We will look into the Native Client and RockSalt
      codebases, and choose one as a starting point.

\item A \textit{Compiler} that outputs code meeting the requirements of our
      verifier. We will use the LLVM \cite{lattner2004llvm} compiler
      infrastructure and plug into its code generation process.

\item A \textit{Memory Manager} (\S \ref{sec:memory_manager}) that effectively
      implements software cache management. The code will be generic enough to
      allow us to target the L1, L2, or L3 cache, and measure the performance
      impact of using either as a scratchpad.

\item \textit{Runtime Glue} that connects the Loader, Verifier and Memory
      Manager into a Trusted Runtime that runs untrusted code and allows it to
      communicate with the outside world.

\end{itemize}

The components are mostly independent of each other, and can be developed in
any sequence. Based on similar previous efforts, we estimate that the Verifier,
Compiler and Memory Manager will each take about a semester (4 months) to be
developed, and the Loader and Runtime Glue will take another semester.

\begin{figure}[hbtp]
  \center{\begin{tabular}{| l | l |}
  \hline
  LLVM Compiler for software attacks & 4 months \\
  \hline
  Verifier for software attacks & 4 months \\
  \hline
  LLVM Compiler for physical attacks & 2 months \\
  \hline
  Verifier for physical attacks & 2 months \\
  \hline
  Memory manager for software attacks & 3 months \\
  \hline
  Memory manager for physical attacks & 3 months \\
  \hline
  Application container format and loader & 4 months \\
  \hline
  OS communication services & 2 months \\
  \hline
  \end{tabular}}
  \caption{
    The tenative schedule for implementing the proposed system. The estimates
    include the time needed to understand the code that we plan to build on,
    the time needed to design and implement the changes, and the time needed
    for benchmarking the implementation.
  }
  \label{fig:cache_partitions}
\end{figure}
